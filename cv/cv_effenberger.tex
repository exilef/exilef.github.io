%%%%%%%%%%%%%%%%%%%%%%%%%%%%%%%%%%%%%%%%%
% Resume/CV
% Felix Effenberger, 2019
% Version 1.0 (2019/01/27)
%
% - Changed to Merriweather font, spacing modifications
% - New education command
% - Compile with pdflatex
%
% Original author:
% Howard Wilson (https://github.com/watsonbox/cv_template_2004) with
% extensive modifications by Vel (vel@latextemplates.com)
%
% License:
% CC BY-NC-SA 3.0 (http://creativecommons.org/licenses/by-nc-sa/3.0/)
%%%%%%%%%%%%%%%%%%%%%%%%%%%%%%%%%%%%%%%%%

%----------------------------------------------------------------------------------------
%	PACKAGES AND OTHER DOCUMENT CONFIGURATIONS
%----------------------------------------------------------------------------------------

\documentclass[10pt]{article} % Default font size

%----------------------------------------------------------------------------------------
%	PACKAGES AND OTHER DOCUMENT CONFIGURATIONS
%----------------------------------------------------------------------------------------

% geometry
\usepackage[a4paper, hmargin=25mm, vmargin=30mm, top=20mm]{geometry}

% font
\usepackage[utf8]{inputenc}
\usepackage[T1]{fontenc}
\usepackage[light]{merriweather}

\usepackage{setspace}
\setstretch{1.1}

\usepackage{fancyhdr}
\usepackage{lastpage}

% suppress section numbering
\setcounter{secnumdepth}{0} 

% header color
\usepackage{xcolor}
\definecolor{headercolor}{rgb}{0,0,0}

% custom headers
\usepackage{sectsty}
\sectionfont{\color{headercolor}}

% headers and footers
\fancypagestyle{plain}{\fancyhf{}\cfoot{\thepage\ of \pageref*{LastPage}}}
\pagestyle{plain}
\renewcommand{\headrulewidth}{0pt}
\renewcommand{\footrulewidth}{0pt}

% hyperlinks
\usepackage{hyperref}
\hypersetup{
  pdfauthor={Felix Effenberger},
  pdfsubject={Resume},
  pdftitle={Resume},
  pdfkeywords={Resume, CV},
  colorlinks=false,
  urlbordercolor={1 1 .3},
  linkbordercolor={1 1 1},
}

% no parindent
\setlength{\parindent}{0pt}

% non-indenting itemize
\newenvironment{itemize-noindent}
{\setlength{\leftmargini}{0em}\begin{itemize}}
{\end{itemize}}

% left margin
\newlength{\marginwidth}
\setlength{\marginwidth}{2.2cm}

% text width for tabbing environments
\newlength{\smallertextwidth}
\setlength{\smallertextwidth}{\textwidth}
\addtolength{\smallertextwidth}{-\marginwidth}

% square bullet
\newcommand{\sqbullet}{~\vrule height 1ex width .8ex depth -.2ex} % Custom square bullet point definition

\newcommand{\link}[1]{\texttt{\href{#1}{#1}}}

%----------------------------------------------------------------------------------------
%	MAIN HEADER COMMAND
%----------------------------------------------------------------------------------------

\renewcommand{\title}[1]{
{\huge{\color{headercolor}\textbf{#1}}}\\
\rule{\textwidth}{0.5mm}\\
}

%----------------------------------------------------------------------------------------
%	EDUCATION COMMAND
%----------------------------------------------------------------------------------------

\newcommand{\education}[4]{
\begin{tabbing}%
\hspace{\marginwidth}\=\kill%
{#1} \> \textbf{#2}\\
\>\+ \textit{#3}\\[5pt]
\begin{minipage}{\smallertextwidth}
\vspace{5pt}
#4
\end{minipage}
\end{tabbing}
}

%----------------------------------------------------------------------------------------
%	JOB COMMAND
%----------------------------------------------------------------------------------------

\newcommand{\job}[5]{
\begin{tabbing}
\hspace{\marginwidth} \= \kill
{#1} \> \textbf{#3}\\
{#2} \>\+ \textit{#4}\\[5pt]
\begin{minipage}{\smallertextwidth}
\vspace{5pt}
#5
\end{minipage}
\end{tabbing}
}

%----------------------------------------------------------------------------------------
%	SKILL GROUP COMMAND
%----------------------------------------------------------------------------------------

\newcommand{\skillgroup}[2]{
\begin{tabbing}
\hspace{5mm} \= \kill
\sqbullet \>\+ \textbf{#1}\\
\begin{minipage}{\smallertextwidth}
\vspace{5pt}
#2
\end{minipage}
\end{tabbing}
}

%----------------------------------------------------------------------------------------
%	INTERESTS GROUP COMMAND
%-----------------------------------------------------------------------------------------

\newcommand{\interestsgroup}[1]{
\begin{tabbing}%
\hspace{5mm}\=\kill%
#1%
\end{tabbing}
\vspace{-10mm}
}

\newcommand{\interest}[1]{\sqbullet \> \textbf{#1}\\[5pt]}

%----------------------------------------------------------------------------------------
%	TABBED BLOCK COMMAND
%----------------------------------------------------------------------------------------
\newcommand{\tabbedblock}[1]{
\begin{tabbing}%
\hspace{\marginwidth}\=\hspace{\smallertextwidth}\=\kill%
#1%
\end{tabbing}
}

%----------------------------------------------------------------------------------------
%	TAB ITEM
%----------------------------------------------------------------------------------------
\newcommand{\tabitem}[2]{%
#1\>\begin{minipage}[t]{\smallertextwidth}#2\end{minipage}\\
}

%----------------------------------------------------------------------------------------
%	PUBLICATION COMMAND
%----------------------------------------------------------------------------------------
\newcommand{\publication}[5]{
\tabitem
{#1}
{%
  \href{#5}{#2}\\
  #3\\
  #4\\
}
}

%----------------------------------------------------------------------------------------
%	REFERENCE COMMAND
%----------------------------------------------------------------------------------------
\newcommand{\reference}[4]{
\parbox[t]{0.5\textwidth}{
\begin{tabbing}
\hspace{2cm} \= \hspace{4cm} \= \kill
\textbf{Name} \> #1\\
\textbf{Company} \> #2\\
\textbf{Position} \> #3\\
\textbf{Contact} \> \href{mailto:{#4}}{#4}
\end{tabbing}
}}

%----------------------------------------------------------------------------------------

\begin{document}

%----------------------------------------------------------------------------------------
%	NAME AND CONTACT INFORMATION
%----------------------------------------------------------------------------------------

\title{Felix Effenberger -- Résumé}

%

%----------------------------------------------------------------------------------------
%	PERSONAL PROFILE
%----------------------------------------------------------------------------------------

\section{Personal Profile}
%
I am a Mathematician (PhD) and computer scientist by training, turned neuroscientist, turned entrepreneur. 
I am located in Germany and work for my own startups and as freelance researcher / software engineer / data scientist / trainer as well.
My research interests are in (computational) neuroscience, (discrete) mathematics and (high-dimensional) data analysis.
I am also interested in creating open source software.

%----------------------------------------------------------------------------------------
%	EDUCATION SECTION
%----------------------------------------------------------------------------------------

\section{Education}
%
\education
{2007-2011}
{PhD Mathematics \textmd{(summa cum laude)}}
{\href{https://uni-stuttgart.de}{University of Stuttgart, Germany}}
{
Thesis title: \textit{Hamiltonian Submanifolds of Regular Polytopes}\\
Advisor: \href{http://www.igt.uni-stuttgart.de/LstDiffgeo/Kuehnel/}{Prof.~Wolfgang~Kühnel}\\[5pt]
%
Research in fields of discrete topology, geometry and combinatorics.
Grant by German Research Foundation (DFG), Project Ku 1203/5.
Published several papers, attended international conferences and authored open source software \href{https://github.com/simpcomp-team/simpcomp}{simpcomp}.
Scientific visit (1 month) at \href{https://www.cornell.edu}{Cornell University}, Ithaca, NY. 
%collaboration with \href{http://pi.math.cornell.edu/~ebs/}{Prof. Edward Swartz}.
}

\vspace{2mm}

\education
{2002-2007}
{Diploma Mathematics and Computer Science (MSc. equivalent) \textmd{(w/ distinction)}}
{\href{https://uni-stuttgart.de}{University of Stuttgart, Germany}}
{
Thesis Title: \textit{Topology-based Vector Field Visualization on 2-Manifolds}\\
Advisor: \href{https://www.vis.uni-stuttgart.de/institut/mitarbeiter/Weiskopf-00001/}{Prof. Daniel Weiskopf}\\[5pt]
%
Areas of study: pure and applied mathematics (analysis, algebra, geometry, topology, statistics, numerical mathematics), 
computing science (algorithm design, databases, scientific visualization).
Published paper, scientific visit (9 months) at \href{https://www.sfu.ca/}{Simon Fraser University}, Burnaby, BC, Canada. 
%collaboration with \href{https://www.vis.uni-stuttgart.de/institut/mitarbeiter/Weiskopf-00001/}{Prof. Daniel Weiskopf}.
}


%----------------------------------------------------------------------------------------
%	EMPLOYMENT HISTORY SECTION
%----------------------------------------------------------------------------------------

\section{Experience}
%
\job
{Sep 2017 -}{Present}
{Co-Founder and Chief Technology Officer (CTO)}
{Stealth Silicon Valley Startup\textup{, San Francisco, CA}}
{
Worked on neuroscience-inspired signal processing with focus on image and video compression. 
Responsible for everything tech, managed team of 5 engineers. Scrum master, chose technologies and
set coding standards, managed cloud infrastructure, did code reviews. Deep dives into engineering
problems where necessary. Raised angel investments and 3m seed round.\\[5pt]
\textbf{Technologies:} Python, C, C++, Assembly, CUDA, OpenCL, ObjectiveC, Gitlab, Amazon EC2, Microsoft Azure
}

% double affiliation, no template here
\begin{tabbing}
  \hspace{\marginwidth} \= \kill
  {Sept 2015 -} \> \textbf{Postdoctoral Researcher}\\
  {Sep 2017} \> \textit{\href{https://fias.institute}{Frankfurt Institute for Advanced Studies}}, Frankfurt, Germany\\
  \>\+ \textit{\href{http://www.esi-frankfurt.de}{Ernst Strüngmann Institute}}, Frankfurt, Germany\\[5pt]
  \begin{minipage}{\smallertextwidth}
    Postdoctoral Advisor: \href{https://www.fias.science/de/neurowissenschaften/gruppen/hermann-cuntz/}{Dr. Hermann Cuntz}\\
    Research in neuronal morphology, modeling and data analysis. Published several papers and developed
    open source software \href{https://www.treestoolbox.org/}{\texttt{TREES toolbox 2}}.\\[5pt]
    \textbf{Technologies:} Python, Matlab, LaTeX  
  \end{minipage}
\end{tabbing}

\job
{May 2013 -}{Oct 2015}
{Freelance Software Developer}
{\href{https://nextbike.com}{nextbike GmbH}\textup{, Leipzig, Germany}}
{
  Developed data-driven Android application for service staff of bike sharing service.\\
  Contact: \href{https://www.linkedin.com/in/johannes-vockeroth-8885709a/}{Johannes Vockeroth, CTO}\\[5pt]
  \textbf{Technologies:} Android, Java
}
  
\job
{Jan 2013 -}{Jan 2016}
{Co-Founder and Full Stack Developer}
{\href{https://www.modelogiq.com}{modelogiq GmbH}\textup{, Frankfurt, Germany}}
{
  Python and Clojure backend developer and JavaScript frontend developer for fintech startup.\\[5pt]
  \textbf{Technologies:} Python, Clojure, JavaScript
}

\job
{Nov 2011 -}{Sep 2015}
{Postdoctoral Researcher}
{\href{https://www.mis.mpg.de}{Max-Planck-Institute for Mathematics in the Sciences}\textup{, Leipzig, Germany}}
{
  Postdoctoral Advisor: \href{https://www.mis.mpg.de/de/jjost/juergen-jost.html}{Prof. Jürgen Jost}\\
  Research in mathematical neurobiology and computational neuroscience, focus on processes of self-organization
  in cortical neural networks and the fundamentals of learning (synaptic plasticity). Modeling and analysis
  of spiking neuron data. Published several research papers, a book chapter, and developed software
  \texttt{\href{https://github.com/team-hdnet/hdnet}{hdnet}}.\\[5pt]
  \textbf{Technologies:} Python, LaTeX
}

\job
{Jun 2004 -}{Oct 2007}
{Research assistant}
{\href{http://www.igt.uni-stuttgart.de}{University of Stuttgart, Institute of Geometry and Topology}\textup{, Stuttgart, Germany}}
{
  Research in fields of discrete topology, discrete geometry, and combinatorics under grant
  of the German Research Foundation (DFG), Project Ku 1203/5: ``Automorphism groups in combinatorial topology''.
}

%----------------------------------------------------------------------------------------
%	TEACHING
%----------------------------------------------------------------------------------------

\section{Teaching}

\textbf{Max-Planck-Institute for Mathematics in the Sciences}, Leipzig, Germany
%
\tabbedblock{
  %
  \tabitem
  {2014}
  {Seminar: \textit{High-dimensional data analysis}}
  %
  \tabitem
  {2014}
  {Lecture: \textit{Self-organization in computational neuroscience} (joint with Anna Levina)}
  %
  \tabitem
  {2013}
  {Lecture: \textit{An Introduction to Computational Neuroscience}}
}

\textbf{University of Stuttgart}, Stuttgart, Germany
%
\tabbedblock{
  %
  \tabitem
  {2011}
  {Lecture: \textit{Geometry} (assisting Prof. E. Teufel)}
  %
  \tabitem
  {2010}
  {Lecture: \textit{Computer Mathematics} (assisting Prof. H. Harbrecht)}
  %
  \tabitem
  {2010}
  {Lecture: \textit{Programming in C} (assisting Prof. H. Harbrecht)}
  %
  \tabitem
  {2007}
  {Lecture: \textit{Introduction to Algebra and Geometry} (assisting Prof. W. Kimmerle)}
  %
}

\textbf{Summer schools}
%
\tabbedblock{
  %
  \tabitem
  {2015}
  {Lecturer at \href{https://crcns.org/course}{\textit{Berkeley Summer Course in Mining and Modeling of Neuroscience Data}},\\UC Berkeley, CA, USA}
  %
  \tabitem
  {2014}
  {Lecturer and tutor at \href{http://www.neurobiotech.ru/ru/dataNS}{\textit{Data Analysis in Neuroscience}}, Moscow, Russia}
  %
  \tabitem
  {2014}
  {Lecturer at \href{http://sisne.org/previous-editions/lascon-v/?lang=en}{\textit{V Latin American School of Computational Neuroscience}} (LASCON), Natal, Brazil}
}

\textbf{Freelance trainer}
%
\tabbedblock{
%
\tabitem
{2011}
{%
Trainer for intensive course \textit{Introduction to robotics}, 20 hours\\
\textit{\href{https://www.ee-ag.com/}{euro engineering AG}}, Stuttgart, Germany (now Modis)
}
%
\tabitem
{2011}
{%
Trainer for intensive course \textit{Programming C}, 30 hours\\
\textit{\href{https://www.ee-ag.com/}{euro engineering AG}}, Stuttgart, Germany (now Modis)
}
}

%----------------------------------------------------------------------------------------
%	PUBLICATIONS
%----------------------------------------------------------------------------------------

\section{Journal Publications}
%
\tabbedblock{
%
\publication
{2018}
{A regularity index for dendrites - local statistics of a neuron’s input space}
{L.Anton-Sanchez$^{*}$, F.Effenberger$^{*}$, C.Bielza, P.Larrañaga, H.Cuntz\\$^{*}$equal contributions}
{\textit{PLOS Computational Biology} 14(11):e1006593}
{https://journals.plos.org/ploscompbiol/article?id=10.1371/journal.pcbi.1006593}
%
\publication
{2017}
{Universal features of dendrites through centripetal branch ordering}
{A.Vormberg, F.Effenberger, J.Muellerleile, H.Cuntz}
{\textit{PLOS Computational Biology} 13(7):e1005615}
{https://journals.plos.org/ploscompbiol/article?id=10.1371/journal.pcbi.1005615}
%
\publication
{2015}
{Self-organization in balanced state networks by STDP and homeostatic plasticity}
{F.Effenberger, J.Jost, A.Levina}
{\textit{PLOS Computational Biology} 11(9):e1004420}
{http://journals.plos.org/ploscompbiol/article?id=10.1371/journal.pcbi.1004420}
%
\publication
{2015}
{Robust Discovery of Temporal Structure in Multi-neuron Recordings Using Hopfield Networks}
{C.Hillar, F.Effenberger}
{\textit{Procedia Computer Science} 53, 365--374}
{http://www.sciencedirect.com/science/article/pii/S1877050915018165URL}
%
\publication
{2012}
{Simplicial blowups and discrete normal surfaces in \texttt{simpcomp}}
{F.Effenberger, J.Spreer}
{\textit{ACM Communications in Computer Algebra} 45(3/4), 173--176}
{http://dl.acm.org/citation.cfm?id=2110176}
%
\publication
{2011}
{Stacked polytopes and tight triangulations of manifolds}
{F.Effenberger}
{\textit{Journal of Combinatorial Theory}, Series A, 118(6), 1843--1862}
{http://www.sciencedirect.com/science/article/pii/S0097316511000537}
%
\publication
{2011}
{\texttt{simpcomp}: a GAP toolbox for simplicial complexes}
{F.Effenberger, J.Spreer}
{\textit{ACM Communications in Computer Algebra}, 44(3/4), 186--189}
{http://dl.acm.org/citation.cfm?id=1940516}
%
\publication
{2010}
{Hamiltonian submanifolds of regular polytopes}
{F.Effenberger, W.Kühnel}
{\textit{Discrete \& Computational Geometry} 43(2), 242--262}
{http://link.springer.com/article/10.1007/s00454-009-9151-9}
%
\publication
{2010}
{Finding and classifying critical points of 2d vector fields: a cell-oriented approach using group theory}
{F.Effenberger, D.Weiskopf}
{\textit{Computing and Visualization in Science} 13(8), 377--396}
{http://link.springer.com/article/10.1007/s00791-011-0152-x}
%
%\publication
%{YEAR}
%{TITLE}
%{AUTHORS}
%{REF}
%{URL}
}

%----------------------------------------------------------------------------------------
%	BOOK CHAPTERS
%----------------------------------------------------------------------------------------
\vspace*{-8mm}
\section{Book Chapters}

\tabbedblock{
  %
  \publication
  {2015}
  {Discovery of Salient Low-Dimensional Dynamical Structure in Neuronal Population Activity}
  {F.Effenberger, C.Hillar}
  {In \textit{International Workshop on Similarity-Based Pattern Recognition (SIMBAD)}, Springer International}
  {http://link.springer.com/chapter/10.1007/978-3-319-24261-3_16}
  %
  \publication
  {2013}
  {A Primer on Information Theory with Applications to Neuroscience}
  {F.Effenberger}
  {In \textit{Computational Medicine in Data Mining and Modeling}, Springer New York}
  {http://link.springer.com/chapter/10.1007/978-1-4614-8785-2_5}
}

%----------------------------------------------------------------------------------------
%	SOFTWARE
%----------------------------------------------------------------------------------------
\vspace*{-8mm}
\section{Software}
%
{%
  \texttt{TREES2} -- TREES toolbox 2, a neuronal morphology Matlab toolbox.\\
  Joint work with H.Cuntz\\[5pt]
  Website: \link{http://treestoolbox.org}\\
  GitHub: \link{https://github.com/treestoolbox/treestoolbox}\\
}\\[8pt]
%
{%
  \texttt{hdnet} -- Hopfield denoising network.\\
  Joint work with C.Hillar\\[5pt]
  GitHub: \link{https://github.com/team-hdnet/hdnet}\\
  Documentation: \link{http://team-hdnet.github.io/hdnet}\\
}\\[8pt]
%
{%
\texttt{simpcomp} -- a GAP toolbox for simplicial complexes.\\
Joint work with J.Spreer.\\
GAP \textit{shared package} (peer reviewed), 2013.\\[5pt]
GAP repository: \link{http://www.gap-system.org/Packages/simpcomp.html}\\
GitHub: \link{https://github.com/simpcomp-team/simpcomp}\\
Documentation: \link{https://simpcomp-team.github.io/simpcomp}\\
}%\\[8pt]

%----------------------------------------------------------------------------------------
%	SERVICE
%----------------------------------------------------------------------------------------

\vspace*{-2mm}
\section{Professional Service}
%
\skillgroup{Reviewer}
{
\textit{\href{http://www.cosyne.org}{COSYNE Conference}},
\textit{\href{https://link.springer.com/journal/454}{Discrete and Computational Geometry}},
\textit{\href{https://www.journals.elsevier.com/european-journal-of-combinatorics}{European Journal of Combinatorics}},
\textit{\href{https://www.journals.elsevier.com/journal-of-combinatorial-theory-series-a}{Journal of Combinatorial Theory, Series A}},
\textit{\href{https://www.nature.com/srep/}{Nature Scientific Reports}},
\textit{\href{https://www.journals.elsevier.com/neural-networks}{Neural Networks}},
\textit{\href{https://journals.plos.org/ploscompbiol/}{PLOS Computational Biology}},
\textit{\href{https://journals.plos.org/plosone/}{PLOS One}}
}


%----------------------------------------------------------------------------------------
%	LANGUAGES
%----------------------------------------------------------------------------------------

\section{Skills}

\skillgroup{Languages}
{
  \textit{German}: native\\
  \textit{English}: full professional proficiency\\
  \textit{French}: professional working proficiency\\
  \textit{Italian}: elementary proficiency\\
}


%----------------------------------------------------------------------------------------
%	IT/COMPUTING SKILLS SECTION
%----------------------------------------------------------------------------------------

\skillgroup{Programming Languages}
{
\textit{Assembly}, \textit{C}, \textit{C++},
\textit{Clojure},
\textit{CUDA},
\textit{Go},
\textit{Java},
\textit{JavaScript},
\textit{Matlab},
\textit{ObjectiveC},
\textit{OpenCL},
\textit{Perl},
\textit{PHP},
\textit{Python},
\textit{R},
\textit{Shell scripting},
\textit{SQL}
}

%------------------------------------------------

\skillgroup{Miscellaneous}
{
\textit{Systems administration in UNIX/Linux environments},
\textit{DVCS} (Git, Mercurial), 
\textit{Productivity applications} (LaTeX, office software),
\textit{Cloud computing} (AWS, Azure),
\textit{Infrastructure as code tools} (Terraform, Ansible),
\textit{Agile development methodologies}
}

%----------------------------------------------------------------------------------------
%	INTERESTS SECTION
%----------------------------------------------------------------------------------------

%\section{Interests}
%\interestsgroup{
%}

%----------------------------------------------------------------------------------------
%	REFERENCES SECTION
%----------------------------------------------------------------------------------------

%\section{References}

%\reference
%{Bill Lumbergh}
%{Initech Inc.}
%{Vice President}
%{bill@initech.com}
%
%\hfill
%
%\reference
%{Michael "Big Mike" Tucker}
%{Burbank Buy More}
%{Store Manager}
%{mike@buymore.com}

%----------------------------------------------------------------------------------------

\end{document}